\documentclass{problemset}
\usepackage{sectsty}
\usepackage{amssymb,amsfonts}
\usepackage[all,arc]{xy}
\usepackage{enumerate}
\usepackage{mathrsfs}
\usepackage[margin=1in]{geometry}
\usepackage{thmtools}
\usepackage{verbatim}
\usepackage{xltxtra}
\XeTeXlinebreaklocale "ja"
\XeTeXlinebreakskip=0pt plus 1pt
\XeTeXlinebreakpenalty=0
\setmainfont[]{IPAPMincho}
\setsansfont[]{IPAPGothic}

\sectionfont{\fontsize{12}{15}\selectfont}
\author{名前:}
\course{栄養疫学宿題}
%\problemset{PSET}
%\problem{栄養疫学宿題}
%\pageprefix{PROBLEM}
\collab{学籍番号:}

\begin{document}
\section{身体活動レベルⅢの30歳の男性,下記の栄養素1日摂取目標量を計算せよ.}
\begin{itemize}
  \item 鉄   $\rule{3cm}{0.2mm}$ mg
  \item 脂質 $\rule{3cm}{0.2mm}$ g
  \item 糖質 $\rule{3cm}{0.2mm}$ g
  \item 蛋白質 $\rule{3cm}{0.2mm}$g
  \item カルシウム $\rule{3cm}{0.2mm}$ mg
\end{itemize}}

\bigskip
\section{55歳の男性,職場の定期健康診断のあと,職場の医務室を訪れた.身長165cm,体重70kg,健康診断では,肥満以外に特記すべき所見を指摘されなかった.仕事は事務作業で,勤務中はほとんどの時間を机に向かって,過ごしている.通勤には自家用車を使っている.勤務の都合で運動する時間を確保するが難しいため,現在の身体活動レベルのまま,BMIが22となる体重目標に減量することにした.55歳の男性の基礎代謝基準値は21.5 kcal/kg/日である.また,推定エネルギー必要量と基礎代謝の比は,身体活動レベルがⅠ(低い)なら1.50,Ⅱ(ふつう)なら1.75,Ⅲ(高い)なら2.00である.この受診者に提示する一日の推定エネルギー必要量(kcal)を求めよ.計算の過程を記入せよ.}

\end{document}
